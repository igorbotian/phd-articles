\documentclass[12pt,a4paper,oneside]{article}

\title{Разработка методов и алгоритмов классификации и ранжирования документов на основе модификации метода опорных векторов}
\author{Игорь Ботян}
\date{\today}

\usepackage[utf8]{inputenc} % возможность использования Unicode-символов в исходных файлах

\usepackage{datetime} % поддержка календаря и дат
\usepackage{fancyhdr} % использование собственного формата заголовков
\usepackage[T2A]{fontenc} % задание шрифта
\usepackage{geometry} % задания страничных отступов
\usepackage{graphicx} % поддержка PNG-формата в качестве иллюстраций
\usepackage[pdftex,bookmarks,unicode]{hyperref} % поддержка возможности использования ссылок в качестве гиперссылок
\usepackage{setspace} % задание межстрочного интервала
\usepackage{verbatim} % скрытие "comment-окружений" из результирующего документа
\usepackage[english,russian]{babel} % поддержка переносов слов

\hypersetup{
%	bookmarkstype=none,
%	plainpages=false,
	pdftitle={Разработка методов и алгоритмов классификации и ранжирования документов на основе модификации метода опорных векторов},
	pdfauthor={Игорь Ботян},
	pdfsubject={Кандидатская диссертация},
	pdfnewwindow=false,
	pdfkeywords={метод опорных векторов, ранжирование документов, методы классификации документов},
	colorlinks=true,
	linkcolor=black,
	citecolor=black,
	filecolor=black,
	urlcolor=black,
	pageanchor=false} % если пакет "lastpage" импортирован, то необходимо отключить данную опцию

% задание страничных отступов
\geometry{left=2.5cm}
\geometry{right=1.0cm}
\geometry{top=2.0cm}
\geometry{bottom=2.0cm}

\onehalfspacing % задание полуторного межстрочного интервала
\setlength{\parindent}{25pt} % задание абзацного отступа в 5 символов
\setlength{\headheight}{28pt}

\begin{document}

\newcommand{\ClearPageStyle} {%
	\pagestyle{fancy}
	\fancyhf{}
	\renewcommand{\headrulewidth}{0pt}
}

\newcommand{\ApplyCommonPageStyle} {%
	\ClearPageStyle
	\cfoot{\thepage}
}

% =============================================================================

\newcommand{\sub} [1] {%
	\scriptsize{#1}%
}

\newcommand{\Rho}{%
	\mathrm{\rho}%
}

\newcommand{\link} [1] {%
	\textcolor{blue}{#1}%
}

\newcommand{\reference} [1] {%
	\link{[#1]}%
}

\newcommand{\todo} [1] {%
	\textcolor{red}{#1}%
}
\ApplyCommonPageStyle

\begin{titlepage}

\setlength{\parindent}{0pt}

\begin{center}
	\textbf{%
		Министерство образования и науки Российской Федерации\\ %
		Федеральное государственное бюджетное образовательное\\ %
		учреждение высшего профессионального образования\\ %
		``Санкт-Петербургский государственный лесотехнический университет\\ %
		имени С.М. Кирова''
	}
\end{center}

\begin{comment}
Факультет: Механической технологии древесины
Кафедра: Управления и автоматизации производственных процессов
Специальность: 05.13.01 ``Системный анализ, управление и обработка информации (в химико-лесном комплексе)''
\end{comment}

\vspace*{10em}

\begin{center}
\textbf{%
	РАЗРАБОТКА МЕТОДОВ И АЛГОРИТМОВ\\%
	КЛАССИФИКАЦИИ И РАНЖИРОВАНИЯ ДОКУМЕНТОВ\\%
	НА ОСНОВЕ МЕТОДА ОПОРНЫХ ВЕКТОРОВ\\%
}
\end{center}

\vspace*{13em}

\begin{tabbing}
Диссертация\\
\\
Руководитель:\quad\= д.т.н. профессор Уткин Л.В.\\
\\
Санкт-Петербург,\ \today
\end{tabbing}

\vspace*{5em}

Игорь Ботян\\
ул. Евдокима Огнева 10-2-97\\
193230 Санкт-Петербург\\
Тел. +7 967 511 19 64\\
igor.botian@gmail.com

\end{titlepage}

\begin{comment}
Титульный лист является первой страницей диссертации, служит источником информации, необходимой для обработки и поиска документа. 
На титульном листе приводят следующие сведения:
- наименование организации, где выполнена диссертация- статус диссертации - ``на правах рукописи''
- фамилию, имя, отчество диссертанта
- название диссертации
- шифр и наименование специальности
- искомую степень и отрасль науки
- фамилию, имя, отчество научного руководителя или консультанта, учёную степень и учёное звание
- место и год написания диссертации
\end{comment}
\setcounter{page}{2}
\newpage
%\clearpage\pagenumbering{arabic}
%\setcounter{page}{1} 
\setcounter{tocdepth}{3}
\renewcommand{\contentsname}{ОГЛАВЛЕНИЕ}
\tableofcontents

\begin{comment}
Оглавление - перечень основных частей диссертации с указанием страниц, на которые их помещают. 
Заголовки в оглавлении должны точно повторять заголовки в тексте. Не допускается сокращать или давать заголовки в другой формулировке. Последнее слово заголовка соединяют с отточием с соответствующим ему номером страницы в правом столбце оглавления. 
\end{comment}

\newpage
\SectionTitle{ВВЕДЕНИЕ}{introduction}

TODO

\begin{comment}
Введение к диссертации включает в себя следующие основные структурные элементы:
- актуальность темы исследования
- степень ее разработанности
- цели и задачи
- научную новизну
- теоретическую и практическую значимость работы
- методологию и методы исследования
- положения, выносимые на защиту
- степень достоверности и апробацию результатов
\end{comment}
%\begin{comment}
В заключении диссертации излагают итоги выполненного исследования, рекомендации, перспективы дальнейшей разработки темы.
\end{comment}
%\begin{comment}
Сокращение слов и словосочетаний оформляют в соответствии с требованиями ГОСТ 7.11 и ГОСТ 7.12. 
Наличие перечня не исключает расшифровку сокращения и условного обозначения при первом упоминании в тексте. 
Перечень помещают после основного текста. 
Перечень следует располагать столбцом. Слева в алфавитном порядке или в порядке их первого упоминания в тексте приводят сокращения или условные обозначения, справа - их детальную расшифровку. 
Наличие перечня указывают в оглавлении диссертации. 
\end{comment}
%\begin{comment}
При использовании специфической терминологии в диссертации должен быть приведён список принятых терминов с соответствующими разъяснениями. 
Список терминов должен быть помещён в конце текста после перечня сокращений и условных обозначений. 
Термин записывают со строчной буквы, а определение - с прописной буквы. Термин отделяют от определения двоеточием. 
Наличие списка терминов указывают в оглавлении диссертации. 
Список терминов оформляют в соответствии с требованиями ГОСТ Р 1.5. 
\end{comment}
%\begin{comment}
Список литературы должен включать библиографические записи на документы, использованные автором при работе над темой. 
Список должен быть размещён в конце основного текста, после словаря терминов. 
Допускаются следующие способы группировки библиографических записей: алфавитный, систематический (в порядке первого упоминания в тексте), хронологический. 
При алфавитном способе группировки все библиографические записи располагают по алфавиту фамилий авторов или первых слов заглавий документов. Библиографические записи произведений авторов-однофамильцев располагают в алфавите их инициалов. 
При систематической (тематической) группировке материала библиографические записи располагают в определённой логической последовательности в соответствии с принятой системой классификации. 
При хронологическом порядке группировки библиографические записи располагают в хронологии выхода документов в свет. 
При наличии в списке литературы на других языках, кроме русского, образуется дополнительный алфавитный ряд, который располагают после изданий на русском языке. 
Библиографические записи в списке литературы оформляют согласно ГОСТ 7.1. 
\end{comment}
%\begin{comment}
Материал, дополняющий основной текст диссертации, допускается помещать в приложениях. В качестве приложения могут быть представлены: графический материал, таблицы, формулы, карты, ноты, рисунки, фотографии и другой иллюстративный материал. 
Иллюстративный материал, представленный не в приложении, а в тексте, должен быть перечислен в списке иллюстративного материала, в котором указывают порядковый номер, наименование иллюстрации и страницу, на которой она расположена. Наличие списка указывают в оглавлении диссертации. Список располагают после списка литературы. 
Приложения располагают в тексте диссертации или оформляют как продолжение работы на её последующих страницах или в виде отдельного тома. 
Приложения в тексте или в конце его должны иметь общую с остальной частью работы сквозную нумерацию страниц. Отдельный том приложений должен иметь самостоятельную нумерацию. 
В тексте диссертации на все приложения должны быть даны ссылки. Приложения располагают в порядке ссылок на них в тексте диссертации. 
Приложения должны быть перечислены в оглавлении диссертации с указанием их номеров, заголовков и страниц. 
Отдельный том ``Приложения'' должен иметь титульный лист, аналогичный титульному листу основного тома диссертации с добавлением слова ``Приложения'', и самостоятельное оглавление. 
Наличие тома ``Приложения'' указывают в оглавлении первого тома диссертации. 
Приложения оформляют в соответствии с требованиями ГОСТ 2.105.
\end{comment}

\begin{comment}
Диссертация в виде рукописи имеет следующую структуру:
а) титульный лист
б) оглавление
в) текст диссертации
  1) введение
  2) основная часть
  3) заключение
г) список сокращений и условных обозначений*
д) словарь терминов*
е) список литературы
ж) список иллюстративного материала*
з) приложения*
\end{comment}

\end{document}

\begin{comment}
Основной текст должен быть разделен на главы и параграфы или разделы и подразделы, которые нумеруют арабскими цифрами.
Каждую главу (раздел) диссертации начинают с новой страницы.
Заголовки располагают посередине страницы без точки на конце. Переносить слова в заголовке не допускается. Заголовки отделяют от текста сверху и снизу тремя интервалами.
Работа должна быть выполнена печатным способом с использованием компьютера и принтера на одной стороне листа белой бумаги одного сорта формата А4 (210 х 297 мм) через полтора интервала и размером шрифта 12—14 пунктов. Диссертация должна иметь твердый переплет.
Буквы греческого алфавита, формулы, отдельные условные знаки допускается вписывать от руки черной пастой или черной тушью.
Страницы диссертации должны иметь следующие поля: левое — 25 мм, правое — 10 мм, верхнее — 20 мм, нижнее — 20 мм. Абзацный отступ должен быть одинаковым по всему тексту и равен пяти знакам.
Все страницы диссертации, включая иллюстрации и приложения, нумеруются по порядку без пропусков и повторений. Первой страницей считается титульный лист, на котором нумерация страниц не ставится, наследующей странице ставится цифра «2» и т.д.
Порядковый номер страницы печатают на середине верхнего поля страницы.
При наличии нескольких томов в диссертации нумерация должна быть самостоятельной для каждого тома.
Библиографические ссылки в тексте диссертации оформляют в соответствии с требованиями ГОСТ Р 7.0.5.

Иллюстративный материал может быть представлен рисунками, фотографиями, картами, нотами, графиками, чертежами, схемами, диаграммами и другим подобным материалом.
Иллюстрации, используемые в диссертации, размещают под текстом, в котором впервые дана ссылка на них, или на следующей странице, а при необходимости — в приложении к диссертации.
Допускается использование приложений нестандартного размера, которые в сложенном виде соответствуют формату А4.
Иллюстрации нумеруют арабскими цифрами сквозной нумерацией или в пределах главы (раздела).
На все иллюстрации должны быть приведены ссылки в тексте диссертации. При ссылке следует писать слово «Рисунок» с указанием его номера.
Иллюстративный материал оформляют в соответствии с требованиями ГОСТ 2.105.

Таблицы, используемые в диссертации, размещают под текстом, в котором впервые дана ссылка на них или на следующей странице, а при необходимости — в приложении к диссертации.
Таблицы нумеруют арабскими цифрами сквозной нумерацией или в пределах главы (раздела).
На все таблицы должны быть приведены ссылки в тексте диссертации. При ссылке следует писать слово «Таблица» с указанием ее номера.
Перечень таблиц указывают в списке иллюстративного материала. Таблицы оформляют в соответствии с требованиями ГОСТ 2.105.

При оформлении формул в качестве символов следует применять обозначения, установленные соответствующими национальными стандартами.
Пояснения символов должны быть приведены в тексте или непосредственно под формулой.
Формулы в тексте диссертации следует нумеровать арабскими цифрами сквозной нумерацией или в пределах главы (раздела).
Номер заключают в круглые скобки и записывают на уровне формулы справа. Формулы оформляют в соответствии с требованиями ГОСТ 2.105.

\end{comment}