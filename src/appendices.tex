\begin{comment}
Материал, дополняющий основной текст диссертации, допускается помещать в приложениях. В качестве приложения могут быть представлены: графический материал, таблицы, формулы, карты, ноты, рисунки, фотографии и другой иллюстративный материал. 
Иллюстративный материал, представленный не в приложении, а в тексте, должен быть перечислен в списке иллюстративного материала, в котором указывают порядковый номер, наименование иллюстрации и страницу, на которой она расположена. Наличие списка указывают в оглавлении диссертации. Список располагают после списка литературы. 
Приложения располагают в тексте диссертации или оформляют как продолжение работы на её последующих страницах или в виде отдельного тома. 
Приложения в тексте или в конце его должны иметь общую с остальной частью работы сквозную нумерацию страниц. Отдельный том приложений должен иметь самостоятельную нумерацию. 
В тексте диссертации на все приложения должны быть даны ссылки. Приложения располагают в порядке ссылок на них в тексте диссертации. 
Приложения должны быть перечислены в оглавлении диссертации с указанием их номеров, заголовков и страниц. 
Отдельный том ``Приложения'' должен иметь титульный лист, аналогичный титульному листу основного тома диссертации с добавлением слова ``Приложения'', и самостоятельное оглавление. 
Наличие тома ``Приложения'' указывают в оглавлении первого тома диссертации. 
Приложения оформляют в соответствии с требованиями ГОСТ 2.105.
\end{comment}