% sudo apt-get install texlive-latex-base
% sudo apt-get install texlive-fonts-extra
% sudo apt-get install texlive-latex-recommended
% sudo apt-get install texlive-lang-cyrillic

%------------------------------------------------------------------------------

\documentclass[11pt]{beamer}
\usetheme{Boadilla}
\usefonttheme{professionalfonts}
\usefonttheme[onlylarge]{structurebold}
\setbeamertemplate{navigation symbols}{}

%------------------------------------------------------------------------------

\usepackage[utf8]{inputenc} % возможность использования Unicode-символов в исходных файлах

\usepackage{amsmath} % для поддержки расширенных математических символов
\usepackage{amsfonts} % для поддержки математических шрифтов
\usepackage{amssymb} % для поддержки расширенных математических символов

\usepackage[english,russian]{babel} % поддержка переносов слов

%------------------------------------------------------------------------------

\title[]{Общий подход к классификации и ранжированию документов при неточных сравнительных оценках}
\author[И.Ю.~Ботян, Л.В.~Уткин]{И.Ю.~Ботян\andЛ.В.~Уткин}
\institute[СПбГЛТУ]{\large Санкт-Петербургский государственный лесотехнический университет}
\date[SCM 2014]{\large SCM-2014 Saint-Petersburg, 21-23 May 2014}

%------------------------------------------------------------------------------

\newcommand{\Rho}{%
	\mathcal{P}%
}

%------------------------------------------------------------------------------

\begin{document}

%------------------------------------------------------------------------------
\begin{frame}

\titlepage

\end{frame}
%------------------------------------------------------------------------------
\begin{frame}{Введение}

\begin{itemize}
	\item Задача классификации
	\item \(y_0\) и \(y_{max}\) - классы
	\item \(y_0\) - нерелевантность, \(y_{max}\) - наибольшая релевантность
	\item Порядок между ними \(\Rightarrow\) задача ранжирования
	\item Классы определяются экспертами
	\item Эксперты дают свою оценку \(\Rightarrow\) формирование обучающей выборки
	\item Существующие методы работают лишь с простыми оценками
	\begin{itemize}
		\item Оценка точна
		\item Является одиночной (не групповой)
	\end{itemize}
\end{itemize}

\end{frame}
%------------------------------------------------------------------------------
\begin{frame}{Интервальные оценки экспертов}

\begin{itemize}
	\item Но в реальности оценки непростые! Как показывает практика, экспертные оценки зачастую являются или групповыми, или неточными
	\begin{itemize}
		\item Оценка групповая (соответствие ``группа'' - ``группа'')
		\item Оценка неточная
	\end{itemize}
	\item \textbf{Рисунок}
	\item Такие оценки будем называть \emph{интервальными}
	\item Не существует обоснованных методов решения задачи ранжирования на их основе
	\item Цель – разработка такого подхода
\end{itemize}

\end{frame}
%------------------------------------------------------------------------------
\begin{frame}{Стандартная постановка задачи классификации}

\begin{itemize}
	\item Обучающая выборка \\
		\(\mathbf{x}\) – образцы наблюдений, \(z\) - классы
		\[F_{0}(\mathbf{x}, y): (\mathbf{x_1}, y_1), (\mathbf{x_2}, y_2), \cdots, (\mathbf{x_n}, y_n) \in \mathbf{\chi} \times  \{-1, +1\}. \]
	\item Цель классификации - найти функцию \(g(\mathbf{x})\), минимизирующую ошибку классификации
	\item Одно из решений - поиск разделяющей функции \(f(\mathbf{x}, \mathbf{w})\), \\
		\(\mathbf{w}\) - параметры
		\[g(\mathbf{x}) = sgn(f(\mathbf{x}, \mathbf{w})).\]
	\item Как найти \(\mathbf{w}\)? Путём вычисления функционала риска
		\[R(\mathbf{w}) = \int \limits_{\chi \times \{-1, +1\}} L(f, \mathbf{x}, y) \mathrm{d} F_0(\mathbf{x}, y).\]
	\item \(L(f, \mathbf{x}, y)\) - функция потерь \\
		\(0\) - классификация верна, иначе ошибка классификации
\end{itemize}

\end{frame}
%------------------------------------------------------------------------------
\begin{frame}{Задача классификации на основе попарного обучения ранжированию}

\begin{itemize}
	\item Сравниваются пары \((\mathbf{x}, \mathbf{z})\) из множества объектов
	\item Идея Лиу: использование математических ожиданий \\
	Соответствие \((\mathbf{x}, \mathbf{z})\) и \(y \in \{-1, +1\}\) \\
		\(y = 1 \Rightarrow \mathbf{x} \succ \mathbf{z}\), \(y = -1 \Rightarrow \mathbf{z} \succ \mathbf{x}\)
		\[R(\mathbf{w}) = \int \limits_{\chi \times \chi \times \{-1, 1\}} L(f, \mathbf{x}, \mathbf{z}, y) \mathrm{d} F_1(\mathbf{x}, \mathbf{z}, y).\]
	\item Тогда задача классификации - поиск ранжирующей функции \(f(\mathbf{x}, \mathbf{w})\), минимизирующей функционал риска
	\item Для её поиска можно воспользоваться RankSVM
		\[\frac{1}{2}<w, w> + C \cdot \sum \limits_{i=1}^n \xi_i,\]
		при ограничениях \\
		\(f(\mathbf{x_i}, w) - f(\mathbf{z_i}, w) \geq 1 - \xi_i, \xi_i \geq 0, i = 1, \cdots, n.\)
	\item \(f(\mathbf{x}, \mathbf{w})\) - линейна \(\Rightarrow\) задача квадратичного программирования
\end{itemize}

\end{frame}
%------------------------------------------------------------------------------
\begin{frame}{Основные элементы теории Демпстера-Шефера}

\begin{itemize}
	\item \emph{U} – универсальное множество (\emph{фрейм различения}) \\
		\emph{N} наблюдений элемента \(u \in U\) \\
		(неточное измерение) \(\Rightarrow\) множество значений \emph{A}
	\item \(c_i\) - количество появлений \(A_i \subseteq U\)
	\item \(\Rho_o(U)\) - множество всех подмножеств \emph{U}
	\item Базовая вероятность (БВ)
		\[m : \Rho_o(U) \to [0,1], m(\varnothing) = 1, \sum \limits_{A \in \Rho_o(U)} m(A) = 1.\] 
		\[m(A_i) = c_i / N.\]
	\item Математическое ожидание БВ (для \(h(x)\)) (верхнее)
		\[\mathbb{\overline{E}} h = \sum \limits_{i=1}^N m(A_i) \inf_{x \in A_i} h(x).\]
\end{itemize}

\end{frame}
%------------------------------------------------------------------------------
\begin{frame}{Формальная постановка задачи в терминах неточных сравнений (часть 1)}

\begin{itemize}
	\item \(\mathbf{\Psi} = \{\mathbf{x_1}, \mathbf{x_2}, \dots, \mathbf{x_M}\}\) - объекты для сравнения. \\
	Пусть у нас есть \emph{M} объектов для сравнения. \\
	Тогда \(M \cdot (M-1)\) пар для сравнения - \emph{фрейм различения}
	\item Обучающая выборка - \(b \leq N\) оценок вида \(\mathbf{A_i} \succ \mathbf{B_j}\), где \(\mathbf{A}\) и \(\mathbf{B}\) - множества
	\item \(c_{ij}\) - количество оценок \(\mathbf{A_i} \succ \mathbf{B_j}\), где \((i, j) \in \Rho\)
	\item \emph{n} - множество пар индексов \((i, j) \in \Rho\) \\
		Тогда базовая вероятность оценки \(m(\mathbf{A_i} \succ \mathbf{B_j}) = c_{ij} / n\)
	\item Если \(c_{ij} = 1\), то \(m(\mathbf{A_i} \succ \mathbf{B_i}) = 1/n\)
\end{itemize}

\end{frame}
%------------------------------------------------------------------------------
\begin{frame}{Формальная постановка задачи в терминах неточных сравнений (часть 2)}

\begin{itemize}
	\item Тогда функционал риска определяется так:
	\[\overline{R}(\mathbf{w}) = \mathbb{\overline{E}}L = \frac{1}{n} \sum \limits_{(i,j) \in \Rho} \underset{\mathbf{x} \in \mathbf{A_i}, \mathbf{z} \in \mathbf{B_j}}{\operatorname{max}}L(f(\mathbf{x}, \mathbf{w}) - f(\mathbf{z}, \mathbf{w})).\]
	\item Наиболее благоприятный случай: \(\emph{max} \rightarrow \emph{min}\)
\end{itemize}

\end{frame}
%------------------------------------------------------------------------------
\begin{frame}{Минимаксная стратегия (часть 1)}

\begin{itemize}
	\item \textbf{Цель} - минимизация верхней оценки риска над множеством параметров \textbf{w}
	\item Новые переменные оптимизации:
		\[G_{ij} = \underset{\mathbf{x} \in \mathbf{A_i}, \mathbf{z} \in \mathbf{B_j}}{\operatorname{max}} L (f (\mathbf{x}, \mathbf{w}) - f(\mathbf{z}, \mathbf{w})).\]
	\item Задача оптимизации выглядит так:\\
		\[\overline{R}(\mathbf{w}) = \underset{\mathbf{w}, G_{ij}}{\operatorname{min}} \sum \limits_{(i, j) \in \Rho} G_{ij},\] 
		при ограничениях\\
		\(G_{ij} + \langle w, \mathbf{x} - \mathbf{z} \rangle \geq 1,\) \\ 
		\(\forall \mathbf{x} \in \mathbf{A_i}, \forall \mathbf{z} \in \mathbf{B_j}, \) \\
		\(G_{ij} \succ 0, (i, j) \in \Rho\).
\end{itemize}

\end{frame}
%------------------------------------------------------------------------------
\begin{frame}{Минимаксная стратегия (часть 2)}

\begin{itemize}
	\item Петлевая функция потерь:
		\[L(f, \mathbf{x}, \mathbf{z}) = max (0,1 - f(\mathbf{x}, \mathbf{w}) + f(\mathbf{z}, \mathbf{w})).\]
	\item Получаем задачу линейного программирования:
		\[\overline{R}(\mathbf{w}) = \underset{\mathbf{w}, G_{ij}}{\operatorname{min}} \sum \limits_{(i, j) \in \Rho} G_{ij},\]
		при ограничениях \\
		\(G_{ij} + \langle w, \mathbf{x} - \mathbf{z} \rangle \geq 1,\) \\ 
		\(\forall \mathbf{x} \in \mathbf{A_i}, \forall \mathbf{z} \in \mathbf{B_j}, \)\\
		\(G_{ij} \succ 0, (i, j) \in \Rho\).
\end{itemize}

\end{frame}
%------------------------------------------------------------------------------
\begin{frame}{Заключение}

\begin{itemize}
	\item Сформулирован подход, позволяющий классифицировать документы при интервальных экспертных оценках
	\item Основан на применении машины опорных векторов и теории Демпстера-Шефера
	\item Характеризуется пессимистической стратегией принятия решений
	\item Определены функционалы риска
	\item Получены соответствующие задачи оптимизации
\end{itemize}

\end{frame}
%------------------------------------------------------------------------------
\begin{frame}

\begin{center}
{\Large Спасибо за ваше внимание!}
\end{center}

\end{frame}
%------------------------------------------------------------------------------

\end{document}