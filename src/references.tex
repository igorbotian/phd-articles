\begin{comment}
Список литературы должен включать библиографические записи на документы, использованные автором при работе над темой. 
Список должен быть размещён в конце основного текста, после словаря терминов. 
Допускаются следующие способы группировки библиографических записей: алфавитный, систематический (в порядке первого упоминания в тексте), хронологический. 
При алфавитном способе группировки все библиографические записи располагают по алфавиту фамилий авторов или первых слов заглавий документов. Библиографические записи произведений авторов-однофамильцев располагают в алфавите их инициалов. 
При систематической (тематической) группировке материала библиографические записи располагают в определённой логической последовательности в соответствии с принятой системой классификации. 
При хронологическом порядке группировки библиографические записи располагают в хронологии выхода документов в свет. 
При наличии в списке литературы на других языках, кроме русского, образуется дополнительный алфавитный ряд, который располагают после изданий на русском языке. 
Библиографические записи в списке литературы оформляют согласно ГОСТ 7.1. 
\end{comment}